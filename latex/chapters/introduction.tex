\chapter{Introduction}

The interest in the application fields of machine learning has a rising tendency nowadays. Machine learning is a field of artificial intelligence that gives computers the capability to learn from patterns and inferences, instead of explicit instructions. Machine learning can utilize statistical methods to process predefined data sets and to predict future output values for given data inputs. Artificial neural networks are machine learning tools and they allow the creation of intelligent systems. Furthermore Python and its scientific third-party libraries are appropriate platforms that can create intelligent systems and facilitate machine learning tasks.\medskip

This work belongs in the realm of machine learning application research. Three main tasks are being tackled in this paper: data mining for extracting information from the data set, building and testing multi-layer perceptron models, and the inversion of the single element feedforward neural network.\medskip

Since preprocessing the data is a key factor to neural network performance, inversion can be implemented only after a sufficient neural network model has been established. Data mining is a tool for information extraction, processing, representation and summarization. It is designed to extract information from a data set and transform it into a comprehensible structure for further use. The product of data mining is the preprocessed training set, whereat training methods can be completed.\medskip

The used training methods originates from the application field of machine learning, more precisely from regression. Regression is a supervised learning task that is used to predict values of a desired target variable. Regression can fit a function to the training data in order to find the best fitting parameters. It is a long process to train a multi-layer perceptron properly and find the best fitting regression function. \medskip

The inverse function means the reverse of another function in mathematics. Inversion in connection with machine learning have not received much attention since the rise of machine learning. Neural network inversion procedures seek to find one or more input values that can produce a desired output with a fixed set of weights. Single element inversion methods are designed to find one point from the input space as opposed to evolutionary methods which are used to map multiple input points. \medskip

This work aims to implement and analyze the inversion of a single element feedforward neural network. The implementation that is being tackled in this paper suggests a solution for the inversion problem with the utilization of the Williams-Linder-Kindermann inversion. In the algorithm, the inversion problem is set up as an unconstrained optimization problem and solved by gradient descent, similarly to backpropagation.

