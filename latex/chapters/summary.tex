\chapter{Summary}

This study presents three tasks from the application of machine learning: data mining, neural network training and inverting. The presented tasks were performed in Python with the assistance of its third-party libraries. \medskip

Inversion has not received much attention since the rise of machine learning, so the problem was that Python's libraries did not contain any inversion methods. In this paper a solution was given for computing the inversion of a function in the context of neural networks.\medskip

The goal is to perform the inversion, for what a sufficient number of neural network model has to be established. The necessary tasks are being tackled in this paper. Data mining was used for extracting information from the data set. Multi-layer perceptron models were built and tested with numerous parameter combinations to find the best fitting combination which can predict the outputs the best. Then the inversion of the single element feedforward neural network was completed by the Williams-Linder-Kindermann inversion.\medskip

The Online News Popularity dataset, which is a publicly available dataset was used during the whole process from data mining to the inversion. The training was running in parallel on the supercomputer with thousands of parameter combinations of the multi-layer perceptron. After catching the best fitting combination, the WLK inversion was able to make the inversion of the model correctly. \medskip

This paper contains important studies in the field of artificial intelligence with the implementation of inversion. In future plans, this explanation will be sent to Scikit-Learn as a scientific solution of the inversion problem.